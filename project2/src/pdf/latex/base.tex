%%%%%%%%%%%%%%%%%%%%%%%%%%%%%%%%%%%%%%%%%%%%%%%%%%%%%%%%%%%%%%%%%%%%%%%%%%%%%%%%%%%%
% Add your packages here
\documentclass{article}
\usepackage[margin=1in]{geometry} 
\usepackage{amsmath,amsthm,amssymb,amsfonts, fancyhdr, color, comment, graphicx, environ, svg}
\usepackage{xcolor}
\usepackage{mdframed}
\usepackage[shortlabels]{enumitem}
\usepackage{indentfirst}
\usepackage{hyperref}
\usepackage{longtable}
\usepackage{dot2texi}
\usepackage{tikz}
\usetikzlibrary{arrows,shapes,automata}

\hypersetup{
    colorlinks=true,
    linkcolor=blue,
    filecolor=magenta,      
    urlcolor=blue,
}

\usepackage{listings}
\lstset
{ %Formatting for code in appendix
    numbers=left,
    stepnumber=1,
}
\usepackage[shortlabels]{enumitem}

\pagestyle{fancy}


\newenvironment{problem}[2][Exercise]
    { \begin{mdframed}[backgroundcolor=gray!20] \textbf{#1 #2} \\}
    {  \end{mdframed}}

% Define solution environment
\newenvironment{solution}
    {\textit{Solution:}}
    {}

\renewcommand{\qed}{\quad\qedsymbol}

% prevent line break in inline mode
\binoppenalty=\maxdimen
\relpenalty=\maxdimen

%%%%%%%%%%%%%%%%%%%%%%%%%%%%%%%%%%%%%%%%%%%%%
%Fill in the appropriate information below
\lhead{Silvia Calderón, Jorge González,\\ David Valverde}
\rhead{Teoría de Autómatas \\ II Semestre, 2023} 
\chead{\textbf{Proyecto 1: DFA Avanzada}}
%%%%%%%%%%%%%%%%%%%%%%%%%%%%%%%%%%%%%%%%%%%%%

\begin{document}


\begin{center}
  {\LARGE Aútomata Determinístico de Estados Finitos (DFA)}
\end{center}


\section{Definición formal y componentes}

  Un DFA es un quinteto $M = (Q, \Sigma, \delta, q_0, F)$ en donde:

  \begin{itemize}
      \item Q: conjunto finito de estados
      \item $\Sigma$: alfabeto
      \item $\delta$: $Q \times \Sigma \rightarrow Q$ (función de transición entre estados)
      \item $q_0$: estado inicial t.q. $q_0 \in Q$
      \item $F$: conjunto de estados de aceptación donde $F \subseteq Q$\newline
  \end{itemize}

Según lo anterior, los 5 componentes del autómata ingresado son los siguientes:\\

% IMPORTANT NOTE: this base file is incomplete as it is missing the \end{document} line
% and thus it will not compile by itself
